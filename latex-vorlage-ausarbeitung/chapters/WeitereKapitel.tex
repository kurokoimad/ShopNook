\chapter{Methodologie und Systementwurf }

Hier werden über folgende Punkte diskutiert:

\begin{itemize}
	\item Die Entwicklungskonzeption, die wir während der Implementation der Anwendung benutzt haben, sowie die Tools, Technologien und Methodologien. 
	\item Überblick über Spring Boot und seine wichtigste  Funktionalitäten, die auch in dem Projekt angewendet sein werden. 
	\item Detaillierte Erklärung der Struktur und Design der Anwendung (Klassendiagramm, Sequenzdiagramm) 
	\item Datenbanken, UI-Design und unterschiedliche Komponenten des Systems. 
	\item Wie Spring Boot die Entwicklung und Implementation der verschiedenen Module vereinfacht. 
\end{itemize}


\section{Entwurf und Konzeption}\index{Entwurf und Konzeption}

In der Softwareentwicklung spielen Entwurf und Konzeption eine entscheidende Rolle. Diese Phase des Entwicklungsprozesses befasst sich mit der Strukturierung und Planung von Systemen, bevor die eigentliche Implementierung beginnt. 

Ziel ist es, ein klares und verständliches Modell zu erstellen, das die Anforderungen und Funktionen eines Systems abbildet. In diesem Kapitel werden UML und einige Diagrammtypen vorgestellt.

\subsection{UML}\index{UML}

Die Unified Modeling Language (UML) ist ein wesentliches Werkzeug im Bereich Entwurf. UML ist eine standardisierte Modellierungssprache, die eine Vielzahl von Diagrammen bietet, um unterschiedliche Aspekte eines Systems darzustellen. Diese Diagramme helfen dabei, komplexe Systeme zu visualisieren, zu dokumentieren und zu kommunizieren \cite{UML:2023}. 

\subsection{Klassendiagramm}\index{Klassendiagramm}

In diesem Abschnitt stellen wir das Klassendiagramm für die Anwendung vor. Das Diagramm (siehe \ref{fig:class_diagram}) veranschaulicht die wichtigsten beteiligten Klassen, ihre Attribute und die Beziehungen zwischen ihnen.
Das Klassendiagramm dient der Darstellung der Struktur des eCommerce-Systems. Es umfasst die folgenden Klassen:

\begin{itemize}
	\item \textbf{Customer}
	\item \textbf{Order}
	\item \textbf{OrderItem}
	\item \textbf{Address}
	\item \textbf{ProductCategory}
	\item \textbf{Product}
	\item \textbf{State}
	\item \textbf{Country}
\end{itemize}

Die Beziehungen zwischen diesen Klassen sind wie folgt:

\begin{itemize}
	\item Ein \textbf{Kunde} kann mehrere \textbf{Bestellungen} aufgeben.
	\item Eine \textbf{Bestellung} kann mehrere \textbf{Bestellpositionen} enthalten.
	\item Jede \textbf{Bestellposition} ist einem \textbf{Produkt} zugeordnet.
	\item Eine \textbf{Bestellung} hat eine \textbf{Lieferadresse} und eine \textbf{Rechnungsadresse}.
	\item Ein \textbf{Produkt} gehört zu einer \textbf{Produktkategorie}.
	\item Ein \textbf{Bundesstaat} gehört zu einem \textbf{Land}.
\end{itemize}

\begin{figure}[h]
	\centering
	\includegraphics[width=\textwidth]{images/ClassDiagram.png}
	\caption{Klassendiagramm für die E-commerce Anwendung ShopNook}
	\label{fig:class_diagram}
\end{figure}

\subsection{Sequenzdiagramm}

Ein Sequenzdiagramm ist eine Art von UML-Diagramm, das den Austausch von Nachrichten zwischen verschiedenen Objekten in einem System im zeitlichen Verlauf darstellt. Es zeigt, wie die verschiedenen Komponenten eines Systems miteinander interagieren, um eine bestimmte Funktion oder einen bestimmten Prozess auszuführen \cite{Sequenzdiagramm:2021}.
Sequenzdiagramme sind besonders nützlich, um die Kommunikation und die Reihenfolge der Nachrichten in einem Prozess zu visualisieren. Sie eignen sich auch besonders gut für objektorientierte Software, da sie den Kontrollfluss bei Objektinteraktionen darstellen \cite[S.28]{swain2010test}. 
\begin{figure}[h]
	\centering
	\includegraphics[width=\textwidth]{images/SequenceDiagram.jpg}
	\caption{Sequenzdiagramm der Anwendung ShopNook mit OAuth 2.0-Autorisierung}
	\label{fig:Sequence-Diagram} 
\end{figure}

In dem vorliegenden Sequenzdiagramm \ref{fig:Sequence-Diagram} wird der Ablauf einer Autorisierungsanfrage gezeigt. Es stellt dar, wie ein Benutzer über eine Client-Anwendung mit einem Autorisierungsserver (Okta) und einem Ressourcenserver interagiert, um Zugriff auf geschützte Ressourcen zu erhalten. \\
Das Diagramm zeigt die verschiedenen Schritte, wie die Anmeldung des Benutzers, die Überprüfung der Anmeldeinformationen durch den Autorisierungsserver, die Ausgabe eines Tokens und die anschließende Validierung dieses Tokens durch den Ressourcenserver, bevor dem Benutzer der Zugang zu den gewünschten Ressourcen gewährt wird.

\section{Front-End Technologien}\index{Front-End Technologien}

Dieses Kapitel befasst sich mit Front-End-Technologien wie HTML, CSS, Typescript und Angular, die für die Erstellung interaktiver, reaktionsfähiger und visuell ansprechender Benutzeroberflächen in unserer E-Commerce-Anwendung unerlässlich sind.

\subsection{HTML/CSS}\index{HTML/CSS}

HTML and CSS sind die grundlegenden Technologien zur Erstellung und Gestaltung von Webseiten. HTML liefert die Struktur und den Inhalt einer Webseite, während CSS für die visuelle Darstellung, Layout, Farben und Schriftarten, verwendet wird. Sie ermöglichen es optisch ansprechende und gut strukturierte Webseiten zu erstellen \cite{HTML/CSS:2024}.

\subsection{Typescript}\index{Typescript}

TypeScript\footnote{https://www.typescriptlang.org} ist eine stark typisierte Programmiersprache, die auf JavaScript aufbaut und statische Typdefinitionen hinzufügt. Sie soll die Entwicklung umfangreicher Anwendungen überschaubarer machen, indem sie es den Entwicklern ermöglicht, Fehler schon früh im Entwicklungsprozess zu erkennen, wodurch Fehler reduziert und die Codequalität verbessert werden. \cite{Typescript:2024}.\\
Die wichtigsten Vorteile von TypeScript sind:
\begin{itemize}
	\item \textbf{Statische Typisierung:} Die statische Typisierung  ermöglicht es Entwicklern, Variablentypen explizit zu definieren, was zu einer besseren Tooling-Unterstützung führt, z. B. bei der Code-Vervollständigung und beim Refactoring. Es hilft auch bei der Identifizierung von typbezogenen Fehlern während der Kompilierungszeit und nicht zur Laufzeit.
	\item \textbf{Wartbarkeit:} Durch die Hinzufügung von Typdefinitionen wird der Code selbstdokumentierender und leichter verständlich, was in großen Codebasen, in denen mehrere Entwickler zusammenarbeiten, entscheidend ist.
\end{itemize}
Durch die Wahl von TypeScript für das ShopNook-Projekt wurde es sichergestellt, dass das Frontend der Anwendung robust, wartbar und skalierbar ist und die hohen Anforderungen an eine professionelle E-Commerce-Plattform erfüllt.

\subsection{Angular}\index{Angular}

Angular\footnote{https://angular.io} ist ein Open-Source-Framework für Webanwendungen, das von Google entwickelt und gepflegt wird. Es wird für die Erstellung dynamischer, einseitiger Anwendungen (SPAs) mit TypeScript und HTML verwendet. Das Framework bietet eine robuste Plattform für die Entwicklung komplexer Anwendungen, indem es einen umfassenden Satz von Tools und Funktionen wie Datenbindung, Dependency Injection und eine modulare Architektur bietet \cite{Angular:2024}.\\
Eine der Hauptstärken von Angular ist seine komponentenbasierte Struktur, die es Entwicklern ermöglicht, wiederverwendbare UI-Komponenten zu erstellen, die die Wartbarkeit und Skalierbarkeit verbessern. Darüber hinaus bietet Angular leistungsstarke Funktionen wie reaktive Programmierung mit RxJS\footnote{https://rxjs.dev}, Zustandsverwaltung und ein reichhaltiges Ökosystem von Bibliotheken, was es zu einer idealen Wahl für Anwendungen auf Unternehmensebene macht.

\subsubsection{Komponentenbasierte Architektur}

Die komponentenbasierte Architektur von Angular gliedert die Anwendung in kleine, in sich geschlossene Einheiten, die als Komponenten bezeichnet werden. Jede Komponente kapselt einen bestimmten Teil der Benutzeroberfläche (UI) und die damit verbundene Logik \cite{Angular2:2024}.\\
Komponenten sind die wichtigsten Bausteine für Angular-Anwendungen. Wie es in der Abbildung \ref{a1} bezeichnet wurde besteht jede Komponente aus: 

\begin{figure} [h]
	\centering
	\includegraphics[scale=1]{images/Components.png}
	\caption{Beispielhafte Dateistruktur einer Angular-Anwendung mit verschiedenen Komponenten}
	\label{a1}
\end{figure}

\begin{itemize}
	\item einer Komponentendatei \texttt{<component-name>.component.ts}
	\item einer Vorlagedatei \texttt{<component-name>.component.html}
	\item einer CSS-Datei \texttt{<component-name>.component.css}
	\item Einer Datei mit Prüfvorschriften \texttt{<component-name>.component.spec.ts}
\end{itemize}

Dieser modulare Ansatz bringt mit sich mehrere Vorteile wie Wiederverwendbarkeit und Wartbarkeit.

\subsubsection{Single Page Application (SPA)}

Angular eignet sich besonders gut für die Erstellung von Single Page Applications. In einer SPA werden Abschnitte der Benutzeroberfläche anhand von Aktionen der Benutzer dargestellt. Nur der benötigte Abschnitt wird aktualisiert, anstatt eine völlig neue Seite neu zu erstellen. Bei einer herkömmlichen Webanwendung erfolgt bei einer Benutzeraktion eine Neuzeichnung der gesamten Anwendung, wodurch ein „Flash“ auf der Seite entsteht. Anstatt dem Benutzer eine leere weiße Seite zu zeigen, wenn die Ladezeit mehrere Sekunden dauert, kann die SPA zum Beispiel eine Ladeanimation anzeigen und die anderen Inhalte zum Durchblättern verfügbar halten \cite[S.11]{schmiedehausen2018single}.
Dieser Ansatz bietet laut \cite{Angular3:2024} mehrere Vorteile:
\begin{itemize}
	\item \textbf{Verbesserte Benutzerfreundlichkeit:} SPAs bieten eine reibungslosere und schnellere Benutzererfahrung, da nur die notwendigen Teile der Seite aktualisiert werden, anstatt die gesamte Seite neu zu laden.
	\item \textbf{Geringere Serverlast:} Sie entlasten den Server, da weniger Ganzseitenanfragen gestellt werden, was zu einer effizienteren Nutzung der Serverressourcen führt.
\end{itemize}

\section{Back-End Technologien}\index{Back-End Technologien}

In diesem Kapitel werden Back-End-Technologien untersucht, wobei der Schwerpunkt auf Spring Boot liegt, um robuste, skalierbare und sichere serverseitige Logik und APIs für unsere E-Commerce-Anwendung zu erstellen.

\subsection{Java}\index{Java}

Java ist eine weit verbreitete Programmiersprache, die für ihre Plattformunabhängigkeit, Stabilität und umfassende Bibliotheken bekannt ist. Aufgrund ihrer Vielseitigkeit und Leistung wird Java häufig für die Entwicklung von Back-End-Anwendungen verwendet. In der E-Commerce-Anwendung (ShopNook) wird Java genutzt, um eine robuste und skalierbare serverseitige Logik zu gewährleisten  \cite{Java:2024}.

\subsection{REST-API}\index{REST-API}

REST ist ein Architekturstil für die Entwicklung vernetzter Anwendungen. Er basiert auf einem zustandslosen, Client-Server- und cachefähigen Kommunikationsprotokoll, und in fast allen Fällen wird das HTTP-Protokoll verwendet. REST-APIs sind so konzipiert, dass sie einfach, leichtgewichtig und skalierbar sind, was sie zu einer beliebten Wahl für Webdienste macht \cite{REST:2024}.

\subsubsection{Einführung in REST}
Angelehnt an \cite{REST:2024} bieten REST-APIs eine Möglichkeit, über HTTP auf die Funktionalität und Daten einer Anwendung zuzugreifen. Sie folgen den Prinzipien von REST, die auf Ressourcen und deren Repräsentationen basieren. Jede Ressource wird durch eine eindeutige URI identifiziert und kann laut \cite[S.4]{Kornienko_2021} durch standardisierte HTTP-Methoden manipuliert werden:
\begin{itemize}
	\item GET: Abrufen von Ressourcen
	\item POST: Erstellen neuer Ressourcen
	\item PUT: Aktualisieren bestehender Ressourcen
	\item DELETE: Löschen von Ressourcen
\end{itemize}

REST-APIs sind also aufgrund ihrer Architektur und Funktionsweise weit verbreitet und bieten eine Reihe von Vorteilen:

\begin{itemize}
	\item \textbf{Skalierbarkeit:} Durch das stateless Design können REST-APIs leicht skaliert werden. Jeder HTTP-Request enthält alle notwendigen Informationen, um ihn zu verarbeiten, ohne dass der Server den vorherigen Zustand kennen muss.
	\item \textbf{Flexibilität:} REST-APIs sind flexibel und können mit verschiedenen Datenformaten arbeiten. JSON ist das gebräuchlichste Format, da es leichtgewichtig und gut lesbar ist.
	\item \textbf{Interoperabilität:\footnote{Interoperabilität ist die Fähigkeit verschiedener Systeme oder Software, zusammenzuarbeiten und Informationen nahtlos auszutauschen\cite{wiki:listing}.}} REST-APIs nutzen standardisierte HTTP-Methoden, was ihre Interoperabilität mit verschiedenen Clients und Plattformen gewährleistet.
	\item \textbf{Leichte Integration:} REST-APIs lassen sich leicht in bestehende Systeme integrieren, da sie auf bekannten Webstandards basieren.
\end{itemize}

Um die Sicherheit von REST-APIs in der ShopNook-Plattform weiter zu gewährleisten, spielt die Implementierung von HTTPS in Kombination mit SSL/TLS eine entscheidende Rolle, da diese Technologien eine sichere und verschlüsselte Kommunikation zwischen dem Client und dem Server ermöglichen.

\subsection{HTTPS und SSL/TLS}

HTTPS ist die sichere Version von HTTP, dem Protokoll, über das die Daten zwischen dem Browser des Kunden und der Website, mit der er verbunden ist, gesendet werden. Das „S“ am Ende von HTTPS steht für „Secure“, das bedeutet, dass die gesamte Kommunikation zwischen dem Kunden und dem Server verschlüsselt ist \cite{HTTPS:2024}.\\
SSL/TLS sind allerdings Verschlüsselungsprotokolle, die eine sichere Kommunikation über ein Computernetz ermöglichen. SSL war das ursprüngliche Protokoll, aber es wurde zugunsten von TLS, das sicherer und effizienter ist, veraltet \cite{SSL/TLS:2023}. Diese Protokolle bieten mehrere wichtige Sicherheitsfunktionen, die im Folgenden näher erläutert werden:

\begin{itemize}
	\item \textbf{Verschlüsselung der Daten:} SSL und TLS verschlüsseln die zwischen dem Client und dem Server übertragenen Daten und stellt sicher, dass die Daten, selbst wenn sie abgefangen werden, nicht von Unbefugten gelesen werden können.
	\item \textbf{Server-Authentifizierung:} Diese Protokolle überprüfen auch die Identität des Servers und stellen sicher, dass die Clients mit einem legitimen Server und nicht mit einem Betrüger kommunizieren.
	\item \textbf{Datenintegrität:} Sie bieten auch Datenintegrität, indem es sicherstellt, dass die Daten während der Übertragung nicht manipuliert wurden.
\end{itemize}
In der Anwendung wurde eine HTTPS Konfiguration (siehe \ref{HTTPS-Config}) erstellt, um eine sichere Kommunikation zwischen Clients und dem Server zu gewährleisten. Dazu wurde der Server so eingestellt, dass er auf Port 8443 auf HTTPS-Datenverkehr wartet.\\
Der Konfigurationsprozess beinhaltete die Aktivierung der SSL-Unterstützung, die für die Verschlüsselung der über das Netzwerk ausgetauschten Daten unerlässlich ist.\\
Eine Keystore-Datei im PKCS12-Format \verb*|shopnook-keystore.p12| wurde verwendet, in der die für die Verschlüsselung erforderlichen SSL/TLS-Zertifikate sicher gespeichert sind. Der Keystore ist durch ein Passwort geschützt \textbf{(server.ssl.key-store-password=secret)}, so dass der Zugriff auf die kryptografischen Schlüssel nur autorisierten Prozessen vorbehalten ist. \\ 
Innerhalb des Schlüsselspeichers wird das Zertifikat durch den Alias \textbf{shopnook} identifiziert, auf den in der Konfiguration verwiesen wird, um die SSL-Vorgänge des Servers mit dem richtigen Zertifikat zu verknüpfen. Diese Einrichtung stellt sicher, dass alle zwischen dem Client und dem Server übertragenen Daten verschlüsselt werden, was einen robusten Schutz gegen Abhör- und Man-in-the-Middle-Angriffe bietet.\\
Darüber hinaus ermöglicht die Konfiguration eine einfache Anpassung an unterschiedliche Umgebungen. So kann beispielsweise der Port der Anwendung je nach Einsatzumgebung geändert werden, indem die Eigenschaft \textbf{server.port} angepasst wird. Für die QA-Demo-Umgebung ist der Port auf \textbf{9898} eingestellt, der durch Umschalten der Kommentarzeilen aktiviert werden kann.\\
Durch die sorgfältige Konfiguration dieser SSL-Einstellungen stellen wir sicher, dass ShopNook den Industriestandards für Sicherheit entspricht.

\begin{lstlisting}[language=, label=HTTPS-Config, caption={Implementierung von HTTPS und SSL Konfiguration}, ]
	#####
	#
	# HTTPS configuration
	#
	#####
	#Define the web server's listening port for HTTPS traffic
	server.port=8443 
	
	#QA Demo Port
	#server.port=9898
	
	#Activate SSL support to secure the application with HTTPS
	server.ssl.enabled=true
	
	#Specify the key alias used within the keystore for SSL
	server.ssl.key-alias=shopnook
	
	#Keystore path
	server.ssl.key-store=classpath:shopnook-keystore.p12
	
	#Password to unlock the keystore and access the SSL keys
	server.ssl.key-store-password=secret
	
	#format of the keystore file
	server.ssl.key-store-type=PKCS12
\end{lstlisting}



Durch die Implementierung von HTTPS mit SSL/TLS hält sich ShopNook an die Industriestandards für Datensicherheit, um seine Nutzer zu schützen und das Vertrauen zu erhalten. Dies ist ein entscheidender Aspekt des Backends der Anwendung, insbesondere für die Wahrung der Vertraulichkeit und Integrität sensibler Daten bei Transaktionen.

\subsection{Spring Boot}\index{Spring Boot}

Spring Boot ist ein Framework, das auf dem Spring Framework aufbaut und speziell entwickelt wurde, um schnelle, effiziente und skalierbare Anwendungen zu erstellen. Es bietet eine Vielzahl von Features, die den Entwicklungsprozess beschleunigen und vereinfachen, insbesondere für Java-basierte Webanwendungen und Microservices \cite{Spring-Framework:o.J}.

\subsubsection{Warum wurde Spring Boot ausgewählt ?}
Spring Boot wurde im Rahmen dieses Projekts aufgrund der breiten Unterstützung in der Entwicklergemeinde ausgewählt. Hier sind einige der Hauptgründe:
\begin{itemize}
	\item \textbf{Produktivität:} Das Framework ermöglicht es auf das Schreiben von Programmlogik zu konzentrieren, anstatt sich um die Konfiguration von Technologien zu kümmern.
	\item \textbf{Microservices:} Es eignet sich hervorragend für die Entwicklung der Architekturen von Microservices, indem es Server wie Tomcat bietet, sodass die Anwendungen ohne externen Server laufen können.
	\item \textbf{Integration:} Es lässt sich nahtlos in andere Spring-Projekte und eine Vielzahl von Datenbanken integrieren.
\end{itemize}
Das Framework bietet auch viele Funktionalitäten, die es von anderen abheben. Es hat viele weitere Vorteile, die hier genannt werden:

\begin{itemize}
	\item \textbf{Auto-Konfiguration:} Die Autokonfigurationsfunktion von Spring Boot konfiguriert Ihre Anwendung automatisch anhand der Abhängigkeiten, die Sie dem Projekt hinzugefügt haben. Dadurch wird die Notwendigkeit einer manuellen Konfiguration minimiert und die Entwicklung beschleunigt.
	\item \textbf{Eigenständige Anwendungen:} Spring Boot-Anwendungen können als eigenständige Java-Anwendungen verpackt werden, was die Bereitstellung einfacher und konsistenter macht.
	\item \textbf{Produktionstaugliche Funktionen:} Das Framework umfasst zahlreiche produktionsreife Funktionen wie Zustandsprüfungen, Metriken und externalisierte Konfiguration.	
\end{itemize}

\subsubsection{Wie funktionniert Spring Boot ?}

\begin{enumerate}
	\item \textbf{Initial Setup:} Mit Spring Initializr kann man schnell ein neues Spring Boot-Projekt mit allen erforderlichen Abhängigkeiten und Konfigurationen initialisieren.
	\item \textbf{Main Application Class:} Jede Spring Boot-Anwendung hat eine Hauptklasse, die mit @SpringBootApplication annotiert ist. Diese Annotation ist eine Kombination aus @Configuration, @EnableAutoConfiguration und @ComponentScan.
	\begin{lstlisting}[language=Java, caption={Main-Application-Class-Implementierung in Java}\label{mainClass.java}]
@SpringBootApplication
public class SpringBootEcommerceApplication {	
	public static void main(String[] args) {
		SpringApplication.run(SpringBootEcommerceApplication.class, args);
	}
}		
	\end{lstlisting}
	\item \textbf{Application Properties:} Spring Boot ermöglicht eine einfache Konfiguration durch application.properties-Datei. Dadurch wird die Konfiguration externalisiert, was die Verwaltung verschiedener Umgebungen erleichtert.
	\begin{lstlisting}[language=Java, caption={Implementierung von Application.properties Datei}\label{Application.properties}]
spring.datasource.driver-class-name=com.mysql.cj.jdbc.Driver
spring.datasource.url=jdbc:mysql://localhost:3306/full-stack-ecommerce?
useSSL=false&useUnicode=yes&characterEncoding=UTF-8&allowPublicKeyRetrieval=
true&serverTimezone=UTC
spring.datasource.username=username
spring.datasource.password=password
spring.jpa.properties.hibernate.dialect=org.hibernate.dialect.MySQL8Dialect
spring.data.rest.base-path=/api
	\end{lstlisting}
\end{enumerate}

\subsubsection{Spring Security}

Spring Security ist ein umfassendes und hochgradig anpassbares Framework, das speziell für die Verwaltung der Authentifizierung und Autorisierung in Java-Anwendungen entwickelt wurde. Seine Architektur ermöglicht es Entwicklern, Sicherheitsmaßnahmen zu implementieren, die auf die einzigartigen Anforderungen ihrer Anwendungen zugeschnitten sind, sodass es eine bevorzugte Wahl für Enterprise-Level-Lösungen. Das Framework bietet nicht nur wiederverwendbare Module für die Authentifizierung und Autorisierung, sondern umfasst auch Standardschutzmaßnahmen gegen gängige Web-Schwachstellen wie Cross-Site Request Forgery (CSRF) und verschiedene Exploit-Schutzmaßnahmen \cite{Spring-Security:2020}.

Die Stärke von Spring Security liegt in seiner Flexibilität, die eine nahtlose Integration in ein breites Spektrum von Anwendungsfällen ermöglicht. Entwickler können die Funktionen leicht erweitern, um spezifische Sicherheitsanforderungen zu erfüllen, was in den dynamischen Anwendungsumgebungen von heute entscheidend ist. Dieses Anpassungspotenzial birgt jedoch auch Risiken, da unsachgemäße Konfigurationen zu erheblichen Sicherheitsschwachstellen führen können \cite{Spring-Security:2020}.

Zum Schutz der Endpunkte für Kundenbestellungen wurde Spring Security mit OAuth2-Authentifizierung implementiert, wobei JWT zur Gewährleistung eines sicheren Zugriffs verwendet wird. Die Einrichtung umfasste die Konfiguration von CORS zur Verwaltung von Herkunftsübergreifenden Anfragen und die Anpassung von Sicherheitseinstellungen wie die Deaktivierung des CSRF-Schutzes und die Handhabung der Inhaltsaushandlung. Dieser Ansatz gewährleistet robuste Sicherheit, während er verschiedene Inhaltstypen unterstützt und die gemeinsame Nutzung von Ressourcen über verschiedene Ursprünge hinweg effektiv handhabt.

\subsection{Stripe API}
In diesem Teil wird die Stripe-API näher beleuchtet, beginnend mit einem Überblick über ihre grundlegenden Funktionen und Hauptmerkmale. Anschließend wird die Integration der API in unsere Anwendung beschrieben, gefolgt von spezifischen Anwendungsfällen, die die Vielseitigkeit von Stripe in unserem Projekt veranschaulichen. Diese Analyse zeigt, wie Stripe eine nahtlose und sichere Zahlungsabwicklung ermöglicht.
\subsubsection{Überblick}

Die Stripe-API bietet ein umfassendes Paket von Tools für Entwickler zur Integration und Interaktion mit den Diensten von Stripe. Sie wurde nach REST-Prinzipien mit vorhersehbaren, ressourcenorientierten URLs entwickelt. Anfragen werden mit verschlüsseltem Form Körpern gesendet, und Antworten werden im JSON-Format zurückgegeben. Je nach verwendetem API-Schlüssel kann die Stripe-API entweder im Testmodus oder im Live-Modus betrieben werden. Verschiedene Stripe-Konten können je nach Version und benutzerdefinierten Funktionen unterschiedliche API-Funktionen aufweisen. Die API-Referenz enthält Codebeispiele für die Integration in verschiedenen Programmiersprachen \cite{Stripe-API:o.J}.

\subsubsection{Hauptmerkmale der Stripe API}
Zu den wichtigsten Funktionen der Stripe-API gehören:

\begin{itemize}
	\item Die API bearbeitet Anfragen einzeln, ohne Unterstützung für Massenaktualisierungen, pro Anfrage kann nur ein Objekt aktualisiert werden.
	\item Die Funktionalität kann sich mit jeder neuen Version der API ändern, wenn Updates veröffentlicht werden.
	\item Die Stripe-API unterstützt eine breite Palette von Zahlungsmethoden, darunter Karten, digitale Geldbörsen, Bankabbuchungen und -überweisungen sowie Bankumleitungen, zusammen mit Optionen wie „Sofort kaufen, später bezahlen“ und bargeldbasierten Gutscheinen.
	\item Sie bietet Funktionen zur Umsatzoptimierung, einschließlich Tools für Authentifizierung, Autorisierung, Betrugsprävention, Streitfallmanagement und Abgleich.
	\item Stripe bietet auch Berichts- und Analysetools wie Stripe Sigma für detaillierte Einblicke und das Stripe Dashboard für das Zahlungsmanagement \cite{Stripe-API:o.J}.
\end{itemize}

\subsubsection{Stripe API Integration für nahtlose Zahlungsabwicklung}

\textbf{Integration:} \\


In der Anwendung wurde die Stripe-API sowohl in das Frontend als auch in das Backend integriert, um eine sichere und effiziente Zahlungsabwicklung zu ermöglichen. Auf dem Frontend wurde die Stripe API Version „8.179“ in der Visual Studio Code Umgebung installiert. Der \texttt{stripePublishableKey} wurde eingebunden, um die Funktionalität von Stripe im Frontend zu aktivieren. Außerdem wurde die Stripe.js-Bibliothek durch Hinzufügen des folgenden Skripts eingebunden:

\begin{lstlisting}[language=HTML, basicstyle=\ttfamily\small, frame=single, backgroundcolor=\color{lightgray}]
	<script src="https://js.stripe.com/v3" async></script>
\end{lstlisting}

sodass die Anwendung direkt über den Browser mit der API von Stripe interagieren kann.

Am Backend wurde der geheime Schlüssel von Stripe(stripe.key.secret) sicher in der Datei application.properties der Spring Boot-Anwendung gespeichert. Dieser Schlüssel ist für die sichere Handhabung sensibler Vorgänge wie die Erstellung von Zahlungsintentionen und die Verwaltung von Transaktionen unerlässlich. Die Payment Intent API wurde im Backend zur Abwicklung des Zahlungsprozesses verwendet, um sicherzustellen, dass die Transaktionen authentifiziert, autorisiert und effizient verarbeitet werden \cite{Stripe-Intents:o.J}. \\


\noindent\textbf{Anwendungsfälle:} \\


Die Stripe API spielt eine entscheidende Rolle in verschiedenen zahlungsbezogenen Szenarien innerhalb der Anwendung. Sie wird hauptsächlich verwendet für:
\begin{itemize}
	\item \textbf{Zahlungsabwicklung:} Die Stripe-API wird verwendet, um Zahlungen sicher zu verarbeiten. Dazu gehört das Erstellen von Zahlungsabsichten im Backend mit Java in Spring Boot, das den gesamten Zahlungslebenszyklus verwaltet, von der Zahlungsauslösung bis zur Bestätigung.
	
	\item \textbf{E-Mail-Bestätigung:} Nach einem erfolgreichen Kauf wird die Stripe-API so konfiguriert, dass sie Bestätigungs-E-Mails an die Benutzer sendet, um ihnen eine Aufzeichnung ihrer Transaktion zu liefern.
	
	\item \textbf{Frontend-Zahlungsabwicklung:} Das Frontend interagiert für die Zahlungsabwicklung mit Stripe und nutzt die Stripe.js-Bibliothek, um die Zahlungsinformationen sicher zu sammeln und mit Token zu versehen, bevor sie zur weiteren Verarbeitung an das Backend gesendet werden.
\end{itemize}

Diese umfassende Integration der Stripe-API sowohl in das Frontend als auch in das Backend stellt sicher, dass alle zahlungsrelevanten Vorgänge sicher, effizient und mit einer nahtlosen Benutzererfahrung abgewickelt werden. 

\section{Datenbankstruktur}\index{Datenbankstruktur}
In diesem Abschnitt wird die Datenbankstruktur der Webanwendung untersucht, um ein umfassendes Verständnis für die Organisation, den Zugriff und die Verwaltung der Daten zu vermitteln. Der Schwerpunkt liegt auf dem relationalen Datenbankmodell und den spezifischen Technologien und Frameworks, die zur Handhabung der Datenpersistenz und der Transaktionen verwendet werden. Dabei wird MySQL als das gewählte Datenbanksystem zusammen mit Spring Data JPA für eine nahtlose Dateninteraktion untersucht. In den folgenden Unterabschnitten werden diese Aspekte im Detail behandelt, wobei relationale Datenbanken, die Auswahl von MySQL, Datenpersistenz und Transaktionsmanagement sowie die Integration von MySQL mit Spring Boot behandelt werden.
\subsection{Übersicht über relationale Datenbanken und Auswahl von MySQL}
Die Daten in einer relationalen Datenbank sind in Tabellen mit Spalten und Zeilen organisiert, und auch die Primär- und Fremdschlüssel werden verwendet, um die Beziehungen zwischen den Tabellen herzustellen. Dieser Ansatz ist ideal für E-Commerce-Anwendungen, da er komplexe Abfragen und aussagekräftige Daten verarbeiten kann. Produktlieferungen, Auftragstransaktionen und Kundendaten werden alle ordnungsgemäß verwaltet. Die Möglichkeit, Tabellen zu verknüpfen, erleichtert umfassende Analysen, einschließlich der Nachverfolgung von Kaufdatensätzen und der Bestandskontrolle, während die Einhaltung von ACID-Eigenschaften eine zuverlässige Transaktionsverarbeitung gewährleistet, die für die Auftragsabwicklung und das Zahlungsmanagement entscheidend ist. Relationale Datenbanken wie MySQL bieten die Robustheit und Konsistenz, die für einen sicheren und produktiven Online-Shopping-Betrieb erforderlich sind \cite{IBM:o.J}.

\subsection{Datenbankschemaentwurf mit Spring Data JPA}
Ziel dieses Abschnitts ist es, zu erörtern, wie Spring Data JPA den Entwurf und die Verwaltung von Datenbankschemata vereinfacht. Es wird gezeigt, wie Spring Data JPA es Entwicklern ermöglicht, Entitäten und Beziehungen innerhalb eines Datenbankschemas abzubilden und zu verwalten, und wie es die Komplexität des Datenzugriffs durch seine Repository-Schicht abstrahiert.

Spring Data JPA vereinfacht den Entwurf von Datenbankschemata, indem es Entwicklern ermöglicht, Java-Klassen mithilfe von Annotationen wie @Entity, @Id und @OneToMany auf Datenbanktabellen abzubilden. Diese Annotationen übernehmen automatisch die Schemaerstellung und -verwaltung und stellen sicher, dass die Datenbankstruktur mit dem Domänenmodell der Anwendung übereinstimmt \cite{Docs-Spring:o.J, Baeldung:o.J}.

Die Repository-Schicht in Spring Data JPA abstrahiert die Komplexität des Datenzugriffs und bietet integrierte CRUD-Operationen über Schnittstellen wie JpaRepository. Dies macht manuelle SQL-Abfragen überflüssig und rationalisiert die Datenbankinteraktionen, so dass sich die Entwickler auf die Geschäftslogik konzentrieren können \cite{Docs-Spring-JPA-Repo:o.J, Docs-Spring-JPA-QueryMethods:o.J}.

\subsection{Datenpersistenz und Transaktionen}
In diesem Abschnitt wird erläutert, wie Spring Boot und Hibernate die Datenpersistenz handhaben. Der Schwerpunkt liegt dabei auf CRUD-Operationen, der Rolle des EntityManagers und der Interaktion des Hibernate ORM mit MySQL.

CRUD-Operationen werden mit Spring Boot und Hibernate verwaltet. Der EntityManager ist in diesem Prozess von zentraler Bedeutung und ermöglicht das Erstellen, Abrufen, Aktualisieren und Löschen von Entitäten. Er verwaltet auch den Lebenszyklus von Entitäten innerhalb von Transaktionen und stellt sicher, dass Änderungen bei Bedarf ordnungsgemäß übertragen oder zurückgenommen werden  \cite{Baeldung-EntityManager:2024}.


Hibernate ORM wird verwendet, um Java-Objekte auf Datenbanktabellen abzubilden, was die Interaktion mit der MySQL-Datenbank vereinfacht. Dieses ORM-Framework macht manuelles SQL überflüssig und bietet einen objektorientierten Ansatz für Datenbankoperationen \cite{Hibernate:o.J}.


\subsection{MySQL-Workbench und SQL}
In diesem Abschnitt wird die MySQL Workbench als Werkzeug für den Entwurf und die Verwaltung von Datenbanken und SQL als Sprache für die Erstellung von Abfragen und die Verwaltung der Datenbank vorgestellt.

MySQL Workbench wurde zur Visualisierung des Datenbankschemas eingesetzt und ermöglichte eine intuitive Gestaltung und Verwaltung der Datenbankstrukturen. Sie erleichterte die Erstellung und Änderung von Tabellen, Beziehungen und Indizes. Darüber hinaus wurde SQL verwendet, um Abfragen für Tests und die Datenbankverwaltung auszuführen, einschließlich Aufgaben wie Datenmanipulation und Schemaaktualisierungen \cite{mySQL:o.J}.

\subsection{Datenbankverbindung und -konfiguration in Spring Boot}
In diesem Abschnitt wird erläutert, wie die Datenbank innerhalb der Spring Boot-Anwendung verbunden und konfiguriert wird.

Die Datenbankverbindung wird in der Datei \texttt{application.properties} konfiguriert. Zu den wichtigsten Eigenschaften gehören die JDBC-URL, der Name der Treiberklasse und die Anmeldeinformationen, wie es in \hyperref[Application.properties]{Listing~2.2} dargestellt wurde.

\section{Authentifizierungs- und Autorisierungsprotokolle}\index{Authentifizierungs- und Autorisierungsprotokolle}

In der modernen digitalen Landschaft ist die Gewährleistung eines sicheren Zugriffs auf Ressourcen von entscheidender Bedeutung. In diesem Abschnitt werden die wichtigsten Protokolle und Standards zur Verwaltung der Benutzerauthentifizierung und -autorisierung untersucht, darunter JWT, OAuth2 und OpenID Connect.

\subsection{Okta}
Okta verbindet Benutzer mit jeder Anwendung über alle Geräte hinweg. Es ist ein Cloud-basierter Identitätsmanagement-Service, der mit verschiedenen lokalen Systemen kompatibel ist. Okta bietet der IT-Abteilung die Möglichkeit, den Zugriff auf Anwendungen und Geräte über eine sichere, zuverlässige und gut geprüfte Cloud-Plattform zu verwalten, mit einer tiefen Integration in lokale Systeme und Verzeichnisse.

Okta bietet Funktionen wie Provisioning, Single Sign-On (SSO), Active Directory (AD) und LDAP-Integration, zentrales Nutzer-Deprovisioning, Multi-Faktor-Authentifizierung (MFA), mobiles Identitätsmanagement und anpassbare Sicherheitsrichtlinien. Diese Funktionen werden durch das Okta Integration Network (OIN) vereinheitlicht, das umfangreiche Integrationsoptionen bietet und SSO für alle Anwendungen ermöglicht, die Ihre Benutzer benötigen \cite{Okta:o.J}.

\subsection{JWT}

JWT ist ein offener Standard (RFC 7519) für die sichere Übertragung von Informationen in Form eines JSON-Objekts. Diese Informationen sind aufgrund der digitalen Unterzeichnung überprüfbar und vertrauenswürdig. JWTs können mit einem geheimen Schlüssel mit HMAC oder mit einem öffentlichen/privaten Schlüsselpaar mit RSA oder ECDSA signiert werden. Hier sind einige Situationen, in denen JSON-Web-Token von Vorteil sind:

\begin{itemize}
	\item \textbf{Autorisierung:}  JWTs werden in der Regel zur Authentifizierung verwendet. Nach der Anmeldung enthält jede Anfrage das JWT, das den Zugriff auf autorisierte Routen, Dienste und Ressourcen ermöglicht.
	\item \textbf{Informationsaustausch:} Mit JWTs werden Informationen sicher zwischen Parteien übertragen. Signierte JWTs bestätigen die Identität des Absenders und stellen sicher, dass der Inhalt unverändert bleibt, da die Signatur sowohl aus der Kopfzeile als auch aus der Nutzlast abgeleitet wird \cite{JWT:o.J}.
\end{itemize}

\subsection{OAuth2}
OAuth 2.0 ist ein offener Sicherheitsstandard, der es Nutzern ermöglicht, Anwendungen von Drittanbietern begrenzten Zugriff auf ihre Ressourcen zu gewähren, ohne ihre Anmeldedaten weiterzugeben. Dieser Rahmen ermöglicht es Anwendungen, Aktionen im Namen von Nutzern durchzuführen und gleichzeitig sicherzustellen, dass sensible Informationen sicher bleiben. Das Protokoll funktioniert über einen Autorisierungsfluss, der den Erhalt eines Zugriffstokens beinhaltet, den die Client-Anwendung für den Zugriff auf geschützte Ressourcen verwendet. Dieser Prozess erhöht die Sicherheit, indem er die direkte Weitergabe von Benutzeranmeldeinformationen verhindert und das Risiko eines unbefugten Zugriffs minimiert \cite{OAuth2:2023}.

OAuth 2.0 ist jedoch nicht unproblematisch. Sicherheitsbedrohungen wie Phishing-Angriffe, CSRF und Token-Diebstahl können Benutzerdaten gefährden, wenn sie nicht richtig angegangen werden \cite{OAuth2:2023}.

\subsection{OpenID Connect}
OpenID Connect ist ein Authentifizierungsprotokoll, das auf dem OAuth 2.0 Framework (IETF RFC 6749 und 6750) aufbaut. Es rationalisiert die Überprüfung der Benutzeridentität, indem es die von einem Autorisierungsserver durchgeführte Authentifizierung nutzt und Benutzerprofilinformationen in einer RESTful und interoperablen Weise bereitstellt.

OpenID Connect fördert ein Internet-Identitäts-Ökosystem durch einfache Integration, Sicherheits- und Datenschutzfunktionen, breite Client- und Gerätehilfe und die Möglichkeit für jede Einrichtung, als OpenID Provider (OP) zu agieren \cite{OpenId:o.J}.


\section{Entwicklungsumgebung und Versionskontrolle}\index{Entwicklungsumgebung und Versionskontrolle}
In diesem Abschnitt werden die integrierten Entwicklungsumgebungen (IDEs) und Versionskontrollsysteme beschrieben, die im Rahmen des Projekts eingesetzt wurden. Er bietet einen Überblick über die für die Entwicklung ausgewählten IDEs und die Versionskontrollwerkzeuge, die zur effektiven Verwaltung von Codeänderungen eingesetzt wurden.
\subsection{Entwicklungsumgebung}
Dieser Unterabschnitt beschreibt die im Projekt verwendeten Entwicklungsumgebungen, wobei der Schwerpunkt auf IntelliJ IDEA und Visual Studio Code (VS Code) liegt. Er hebt die Funktionen und Vorteile der einzelnen IDEs bei der Unterstützung des Entwicklungsprozesses und der Steigerung der Produktivität hervor.

\subsubsection{IntelliJ IDEA}

IntelliJ IDEA ist bekannt für seinen leistungsstarken Code-Editor, der sich durch eine umfassende Anfangsindizierung beim Verstehen und Verwalten von Code auszeichnet. Diese Funktion ermöglicht es ihm, Fehler in Echtzeit zu erkennen, kontextbezogene Vorschläge zur Code-Vervollständigung anzubieten und sicheres Code-Refactoring durchzuführen, neben anderen Funktionen. Es ist mit einer umfangreichen Suite integrierter Entwicklerwerkzeuge ausgestattet und bietet robuste Unterstützung für verschiedene Spring-Frameworks sowohl in Java als auch in Kotlin. Dazu gehören Spring MVC, Spring Boot, Spring Integration, Spring Security und Spring Cloud. Zu den bemerkenswerten Vorteilen gehören intelligente Codierungshilfe, sofortige Navigation innerhalb des Spring-Codes, integrierte Entwicklungswerkzeuge und erweiterte Visualisierungsfunktionen \cite{Jetbrains:o.J}.


\subsubsection{Visual Studio Code}

 Visual Studio Code ist ein vielseitiger und effizienter Quellcode-Editor, der mit Windows, macOS und Linux kompatibel ist. Er bietet native Unterstützung für JavaScript, TypeScript und Node.js und verfügt über eine breite Palette von Erweiterungen für weitere Sprachen und Plattformen \cite{visual-studio:o.J}.

Da die Front-End-Entwicklung Angular, ein auf TypeScript basierendes Framework, nutzt, wurde Visual Studio Code als Entwicklungsumgebung gewählt. Die starke native Unterstützung für TypeScript erleichtert die effiziente und optimierte Projektentwicklung und macht zusätzliche Erweiterungen überflüssig \cite{visualstudio-angular:o.J}.

\subsection{Versionskontrolle}
In diesem Unterabschnitt werden die im Projekt eingesetzten Versionskontrollsysteme beschrieben, wobei der Schwerpunkt auf Git und GitHub liegt. Es wird erläutert, wie diese Tools die Codeverwaltung, die Zusammenarbeit und die Verfolgung von Änderungen während des gesamten Entwicklungsprozesses erleichtert haben.
\subsubsection{Git}
Git ist ein kostenloses und quelloffenes, verteiltes Versionskontrollsystem, das entwickelt wurde, um Projekte jeder Größe schnell und effizient zu verwalten. Es hat einen minimalen Platzbedarf und liefert eine außergewöhnlich schnelle Leistung. Git übertrifft herkömmliche SCM-Tools wie Subversion, CVS, Perforce und ClearCase mit Funktionen wie kostengünstiger lokaler Verzweigung, benutzerfreundlichen Staging-Bereichen und flexiblen Arbeitsabläufen.

Eines der herausragenden Merkmale von Git ist sein Verzweigungsmodell, das es von fast allen anderen SCM-Tools unterscheidet. Git ermöglicht und fördert die Erstellung mehrerer lokaler Zweige, von denen jeder unabhängig funktionieren kann. Das Erstellen, Zusammenführen und Löschen von Zweigen geht schnell, nahtlos vonstatten und dauert oft nur Sekunden  \cite{git-scmr:o.J}.

\subsubsection{GitHub}
GitHub ist eine cloudbasierte Plattform zum Speichern, Freigeben und gemeinsamen Bearbeiten von Code. Die von Git unterstützten Kollaborationsfunktionen von GitHub ermöglichen es uns:

\begin{itemize}
	\item Projekte zu präsentieren oder zu verbreiten.
	\item Verfolgen und Verwalten von Codeänderungen im Laufe der Zeit.
	\item Andere einladen, den Code zu überprüfen und Verbesserungen vorzuschlagen.
	\item Zusammenarbeit an Projekten ohne das Risiko, dass unbeabsichtigte Änderungen die Arbeit der anderen beeinträchtigen, bis die Änderungen integriert werden können \cite{github:o.J}.
\end{itemize} 














