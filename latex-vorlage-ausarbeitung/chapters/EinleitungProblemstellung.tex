\chapter{Einleitung und Problemstellung}


Hier werden folgende Aspekte berücksichtigt: 


\begin{itemize}
	\item Ein kleiner Überblick über die Anwendung ShopNook
	\item und deren funktionale, nicht funktionale und technische Anforderungen.
\end{itemize}

\section{Überblick}\index{Überblick}

ShopNook ist ein internetbasierter Marktplatz, der ein nahtloses und angenehmes Einkaufserlebnis bieten soll. Käufer können ganz einfach eine breite Palette von Produkten erkunden, sie in den Warenkorb legen und ihre Einkäufe sicher abschließen. Die Plattform verfügt über ein attraktives und reaktionsfähiges Design, das eine konsistente und ansprechende Benutzererfahrung auf allen Gerätegrößen gewährleistet.\\
Neben der benutzerfreundlichen Oberfläche verfügt ShopNook über robuste Funktionen wie ein Bestandsverwaltungssystem, Benutzerkonten mit verschiedenen Berechtigungen und erweiterte Suchoptionen, die den Käufern helfen, genau das zu finden, was sie brauchen. Die Anwendung ist in der Lage, verschiedene administrative Aufgaben zu erledigen und bietet Käufern einen sicheren und effizienten Checkout-Prozess.

\section{Globale Anforderungen}\index{Globale Anforderungen}

\subsection{Funktionale Anforderungen}

Die App wird über alle notwendigen Funktionen verfügen, um ein reibungsloses Einkaufserlebnis zu gewährleisten. Die Kunden werden die Möglichkeit haben, sich sicher zu registrieren, einzuloggen und einen Produktkatalog zu erkunden. Zu jedem Produkt werden vollständige Details wie Name, Beschreibung, Preis und Bilder angezeigt. Die Kunden können schnell Artikel in ihren Einkaufskorb legen, den Inhalt ändern und zur Zahlung übergehen. Sichere Kreditkartenzahlungen werden während des Kaufvorgangs über die Stripe-API abgewickelt. Administratoren können die Waren überwachen und die Bestellungen im Auge behalten, um sicherzustellen, dass das Online-Geschäft gut funktioniert. 

\subsection{Nich -funktionale Anforderungen}

Die App wird effizient mit gleichzeitigen Nutzern umgehen und die Ladezeiten der Seite sind minimal. HTTPS-Verschlüsselung und JWT-basierte Authentifizierung stellen die Sicherheit in den Vordergrund. Die Einheitlichkeit der Geräte wird durch eine reaktionsschnelle und einfach zu bedienende Schnittstelle gewährleistet.

\subsection{Technische  Anforderungen}

Die vollwertige E-Commerce-App wird mit npm für JavaScript-Dependenzen und Maven für die Java Dependenzen und Build-Prozesse entwickelt. Im Frontend wird Angular verwendet, um eine dynamische und reaktionsschnelle Schnittstelle für die Benutzer bereitzustellen, während im Backend das Spring Boot-Framework verwendet wird, um RESTful APIs zu erstellen. MySQL wird die Datenbank sein, die Benutzer-, Produkt- und Bestelldaten speichert. Git und GitHub werden für die Versionskontrolle verwendet, um Änderungen zu verwalten und die Teamarbeit zu erleichtern. 