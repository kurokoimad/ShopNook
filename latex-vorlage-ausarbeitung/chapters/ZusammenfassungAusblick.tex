\chapter{Zusammenfassung und Ausblick}

Im Rahmen dieser Arbeit werden mehrere Schwierigkeiten und Herausforderungen bei der Konzeptentwicklung und Umsetzung der E-Commerce-Warenkorbanwendung „ShopNook'' thematisiert. Zu den Herausforderungen gehören:

\begin{itemize}
	\item \textbf{Technische Integration}: Die nahtlose Integration der verschiedenen Technologien (Angular für das Frontend, Spring Boot für das Backend und MySQL für die Datenbank) stellte eine Herausforderung dar, insbesondere in Bezug auf die Kommunikation zwischen den Komponenten und die Sicherstellung der Datenkonsistenz.
	\item \textbf{Sicherheitsaspekte}: Die Implementierung von Sicherheitsmaßnahmen wie HTTPS-Verschlüsselung und JWT-basierte Authentifizierung erforderte sorgfältige Planung und Tests, um sicherzustellen, dass die Benutzerdaten und Transaktionen geschützt sind.
	\item \textbf{Benutzererfahrung}: Die Gestaltung einer intuitiven und ansprechenden Benutzeroberfläche, die auf verschiedenen Geräten gut funktioniert, war eine weitere Herausforderung. Es war wichtig, ein Gleichgewicht zwischen Funktionalität und Benutzerfreundlichkeit zu finden.
\end{itemize}

Allerdings sind offene Punkte geblieben, die noch weiter untersucht werden müssen, und sie umfassen:

\begin{itemize}
	\item \textbf{Leistungsoptimierung}: Es gilt herauszufinden, wie die Anwendung unter hoher Last performt und welche Maßnahmen zur Optimierung der Ladezeiten und der Reaktionsfähigkeit ergriffen werden können.
	\item \textbf{Erweiterte Funktionen}: Die Implementierung zusätzlicher Funktionen, wie z.B. personalisierte Empfehlungen oder ein verbessertes Bestandsmanagement, könnte die Benutzererfahrung weiter verbessern.
\end{itemize}
